%%%%%%%%%%%%%%%%%%%%%%%%%%%%%%%%%%%%%%%%%
% Friggeri Resume/CV  - modified by Aaron Gonzales
% XeLaTeX Template
%
% Original author:
% Adrien Friggeri (adrien@friggeri.net)
% https://github.com/afriggeri/CV
%
% License:
% CC BY-NC-SA 3.0 (http://creativecommons.org/licenses/by-nc-sa/3.0/)
%
% Important notes:
% This template needs to be compiled with XeLaTeX and the bibliography, if used,
% needs to be compiled with biber rather than bibtex.
%
% Aaron Gonzales
% changed many formatting options, did away with tabular as a style, ditched
% sidebar, changed fonts, margins for the 2nd+ page, etc.
%%%%%%%%%%%%%%%%%%%%%%%%%%%%%%%%%%%%%%%%%

\DocumentMetadata{
 testphase=phase-II,
 pdfversion=2.0
}



\documentclass[print]{friggeri-cv} % Add 'print' as an option into the square bracket to remove colors from this template for printing
\usepackage{microtype}



\usepackage{marvosym}
\usepackage{fancyhdr}
\usepackage{lastpage}
\usepackage{enumitem}
\usepackage[a-2b]{pdfx}
% \usepackage{datetime} % for \pdfdate command
\usepackage[T1]{fontenc}
% \usepackage{lmodern}

\usepackage{fontspec}
\defaultfontfeatures{Mapping=tex-text,Ligatures={NoRequired,NoCommon}}


\usepackage{hyperref}
\hypersetup{% 
    % pdftitle={aaron gonzales resume},
    % pdfauthor={aaron gonzales},
    % pdfkeywords={resume} {aaron gonzales} {machine learning} {technical lead},
    hidelinks,
    pdflang={en},
    pdfstartpage={},
}


\usepackage[document]{ragged2e}


\addbibresource{agbib.bib} % Specify the bibliography file to include publications

\setlist[description]{leftmargin=0.15cm,labelindent=\parindent}

\rfoot{}
%\cfoot{\thinfont{Gonzales CV; page \thepage}}
\cfoot{}
\lhead{}
\chead{}
\rhead{}


\usepackage{etoolbox}
\newtoggle{cv}

% block to take doc args
\ifx\docmode\undefined
  \togglefalse{cv}
\else 
  \toggletrue{cv}
\fi

\begin{document}

\iftoggle{cv}
{\header{Aaron}{~Gonzales}{curriculum vit\ae}
  {\{\href{mailto:aaron@aarongonzales.net}{aaron@aarongonzales.net},
    505.385.9209,
    \href{http://aarongonzales.net}{aarongonzales.net}\}}
}
{\header{Aaron}{~Gonzales}
  %% contact line
  {
    [\href{mailto:aaron@aarongonzales.net}{aaron@aarongonzales.net} |
        505.385.9209 |
        \href{http://lnkd.in/b8kfQSe}{\Large{\faLinkedin}}~|
        \href{http://github.com/binaryaaron}{\huge{\faGithub}}]
  }
  %% summary
  {Machine learning engineer and technical lead who blends responsible machine
    learning, engineering, research, and leadership to accelerate teams.}
}

\section{Experience}

\begin{job}
  {}{Staff Machine Learning Engineer/Architect}{Mozilla.ai}{April 2023}{current}{Remote, USA}{
    Hired as Lead/Founding Engineer to develop Mozilla.ai's foundational research and development platform.
    

    \begin{myitemize}
      \item Established engineering culture and practice, including coding standards, developer tooling, CI/CD
      pipelines, API design styles, abstraction patterns, and a technical hiring program.

      \item Designed and built a platform for large language model (LLM) research and product iteration on a
      cloud-based GPU cluster. Continued with a small team to iterate and enhance the platform.

      \item Managed procurement, setup, and relationships of cloud and IT vendors.
      \item Coordinated Neurips and KDD workshops \autocite{kdd_2023_workshop,neurips_llm_efficiency_2023}.


    \end{myitemize}
  }
\end{job}


\begin{job}
  {}{Senior Machine Learning Research Engineer}{Twitter}{June 2020}{January 2023}{Remote, USA}{
    Technical Lead for the Machine Learning Ethics, Transparency, and Accountability team that owned
    Responsible Machine Learning development and research at Twitter.
    

    \begin{myitemize}

      \item Improved processes, gave technical guidance, mentored other engineers, and led
      organization-wide improvements.

      \item Built a system of feature pipelines and tools for researchers that accelerated critical team
      research by 20x while ensuring more reliable results.

      \item Led projects to build tools into Twitter's internal Tensorflow Extended ecosystem that enabled teams to
      rapidly identify undue biases and weaknesses in model performance.

      \item Helped push adoption of team work, leading to company-wide adoption of our model quality metrics and tools. Presented work in multiple forums.
      \autocite{twitter_htl_racial_bias,twitter_recsys_distributional_inequality,facct_2022_talk,datascience_2022_talk,privacy_enchancing_tech_post}

      \item Continued service on company advisory groups, including efforts to select a new Twitter-wide data processing stack that would 
      lead to 3-5x speedups in key data pipelines.

    \end{myitemize}
  }
\end{job}

\begin{job}
  {}{Senior Data Scientist}{Twitter}{July 2017}{June 2020}{Boulder, CO}{
    Data scientist on several teams, most importantly as tech lead for the Health Data Science team,
    which worked on integrity, combating abuse, identifying fake accounts, and studying misinformation problems.
    

    \begin{myitemize}
      \item Developed tooling, education, and processes that enabled data scientists to expand and streamline analyses 
      with complex and heterogeneous data sources, which were used by hundreds of data scientists, ML, and data
      engineers weekly.

      \item Developed strategy, model, and data pipeline for identifying potentially compromised accounts.

      \item Developed open-source client libraries for Twitter API products,
      which were used by thousands of API customers
      \autocite{search_tweets_python,do_more_with_twitter_data}.

      \item Built risk model for developers wanting to use Twitter APIs, and model for identifying clusters of
      applications abusing Twitter's API rate limits, allowing programmatic remediation.

      \item Served on company advisory groups for data architecture, open-source software, Python development,
      and as a global lead for Twitter's Hispanic/Latinx employees resource group.

    \end{myitemize}
  }
\end{job}

\begin{job}
  {}{Data Scientist}{Tripadvisor}{July 2016}{July 2017}{Greater Boston Area, MA}{
    Built scalable ML models for the Vacation Rentals group.
    \begin{myitemize}
      \item Designed and built a computer vision feature engineering system that
      includes models to detect humans, extract content, and rate aesthetic
      quality of all photo assets, increasing a primary click-through rate
      metric by 4.8\% and increase in revenue from ad expenditures by 10.4\%.

      \item Built a language model and interface that enables natural-language queries (e.g.\ ``girls getaway'') to find
      collections of destinations, which led to a \textbf{6.1\% improvement} in a primary metric and was
      used by many teams.

      \item Developed language-modeling tools to extract high-information snippets from reviews to display on
      search-result pages to boost SEO rankings.

      \item Developed a common team codebase providing shared tools, common access methods, and other utility
      code that allowed data scientists and engineers to get up-to-speed more quickly and iterate
      faster on projects.
    \end{myitemize}
  }
\end{job}

\iftoggle{cv}{
  \begin{description} \itemsep1pt \parskip0pt \parsep0pt
    \item \tripicon {\largeheaderfont Data Scientist Intern, TripAdvisor } \hfill
          {\smallheaderfont Summer 2015}\\*
          {\footnotesize \emph{Greater Boston Area, MA}} \\*
          Data scientist intern with the Vacation Rentals team.  Developed a novel
          ranking system for 720,000+ rental properties and used gradient-boosting
          machines to predict how well new properties will perform. \\*
          \begin{itemize} \itemsep1pt \parskip1pt \parsep0pt

            \item The model scaled with millions of daily visitors and self-tunes to fluctuations in visitor usage
                  patterns.

            \item A/B testing showed that the model \textbf{increased a key visitor conversion rate by 3.46\%},
                  \textbf{decreased visitor bounce rate by 0.46\%}, and \textbf{increase revenue per visitor by
                    9.57\%}.

            \item The model was put into production across all of TripAdvisor Vacation Rentals' sites\\*
          \end{itemize}
  \end{description}
}{}

\iftoggle{cv}{
  \item {\largeheaderfont Research Assistant, The University of New Mexico} \hfill
  {\smallheaderfont 2015\textemdash 2016}\\*
  {\footnotesize \emph{Albuquerque, NM }} \\*
  Research assistant for Dorian Arnold, PhD, in the Scalable Systems Lab. We
  partnered with both Los Alamos National Laboratory and the Center for
  Advanced Research Computing to investigate applying data science techniques
  to understand complex high-performance system behavior. \\*
}{}

\iftoggle{cv}{
  \item {\largeheaderfont Analyst/Programmer, The University of New Mexico} \hfill
  {\smallheaderfont 2011\textemdash 2014 }\\*
  {\footnotesize \emph{Albuquerque, NM }} \\*
  Lead analyst in a neuroscience research lab run by Elaine Bearer,
  MD\textemdash PhD, managing various research projects and lab members. \\*
  \begin{itemize} \itemsep0.5pt \parskip0pt \parsep0pt
    \item Streamlined lab data processing and analytical techniques, including a method that speed up a
          critical data processing step by approximately 360x (3 hours to 30 seconds). Implemented numerous
          other batch data processing steps for other tasks.
    \item Designed and conducted a pilot study that helped win a \$2.7 million dollar NIH R01 grant to study
          the etiology of post traumatic stress disorder.
    \item Lead author on several research papers, three conference abstracts and presentations, coauthor on
          many more submitted and pending papers and abstracts.
    \item Trained and mentored 10 undergraduate and gradute student employees and volunteers.
  \end{itemize}
}{}
%------------------------------------------------
\iftoggle{cv}
{\item  {\largeheaderfont Research Associate} \hfill {\smallheaderfont 2010\textemdash 2011} \\*
  {\footnotesize \emph{The Mind Research Network, Albuquerque, NM}}  \\*
  {\smallheaderfont summary:} \\*
  RA in a neuroscience research group ran by Julia Stephen, PhD. Contributed
  to a study investigating multi-sensory integration in patients with
  schizophrenia. \\*
}

%------------------------------------------------
%\section{research internships}
\iftoggle{cv}{
  \begin{description}
    \item  {\largeheaderfont Student Volunteer } \smallheaderfont{\hfill 2008\textemdash 2010} \\*
          {\footnotesize \emph{The Mind Research Network, Albuquerque, NM}} \hfill \\*
          Assisted with a study ran by Pilar Sanjuan, PhD investigating substance use
          and post-traumatic stress disorder (PTSD) in recently returned combat
          veterans. \\*
  \end{description}

  %%%%%%% sets this to remove the sidebar %%%%%%%%%%
  %\newgeometry{ left=1.75cm,top=2cm,right=1.75cm,bottom=1.50cm,nohead,nofoot }

  \begin{description}
    \item  \largeheaderfont{Student Volunteer}  \hfill {\smallheaderfont 2008 } \\*
          {\footnotesize \emph{The University of New Mexico, Albuquerque, NM}\textemdash } \\*
          Assisted Akaysha Tang, PhD, with a study investigating stress regulation in
          rats and assisted an expert in troubleshooting and repairing an EEG system
          for a study investigating stress in humans.\\*
          %------------------------------------------------
  \end{description}
}{} %%else nothing in the CV section.
%----------------------------------------------------------------------------------------
%	Education
%----------------------------------------------------------------------------------------

\section{Education}
\begin{edu}{Master of Science, Computer Science}{2016}{University of New Mexico}
\end{edu}
\begin{edu}{Bachelor of Science, Psychology}{2010}{University of New Mexico}
\end{edu}

\section{Technical Skills}

\begin{skilltype}
  {Languages}{Python, SQL, shell, Scala}
\end{skilltype}
\begin{skilltype}
  {ML/Data/AI frameworks}{Spark, Tensorflow, Kubeflow, TFX, Pytorch, Pandas, scikit-learn, Keras, vllm, Ray}
\end{skilltype}

\begin{skilltype}
  {Dev and ML Ops frameworks/platforms}{Google Cloud, Coreweave, Oracle Cloud, Kubernetes, Helm, Airflow, Bazel}
\end{skilltype}

\iftoggle{cv}{
  \section{Publications}
}
{
  \section{Selected Publications and Artifacts}
}
\iftoggle{cv}{
  \begin{refsection} % This is a custom heading for those references marked as "inproceedings" but not containing "keyword=france"
    \nocite{*}
    \newrefcontext[sorting=chronological]
    \printbibliography[type=article, title={Articles In Peer-Reviewed Journals}, heading=subbibliography]
    \newrefcontext[sorting=chronological]
    \printbibliography[type=inproceedings, title={Conference Proceedings}, heading=subbibliography]
  \end{refsection}
}
{
  %%%%%%%%%%%%% . else non CV
  \newrefcontext[sorting=chronological]
  \nocite{network_induced_memory_contention,
    % comp_brain_phenotyping_paper,
    skullstrip,do_more_with_twitter_data} \leavevmode\printbibliography[heading=none,
    keyword={selected}] 
}

%----------------------------------------------------------------------------------------
%	INTERESTS SECTION
%----------------------------------------------------------------------------------------

% \section{Other Interests and Accolades}
% \textbf{Olympic Weightlifting}
% \begin{itemize}
% \item 2014 New Mexico Games: Gold Medalist, 94kg class
% \item 2013 New Mexico Games: Silver Medalist, 85kg class
% \item 2013 Barnholth Memorial Invitational: Silver Medalist, 85kg class
% \end{itemize}

\end{document}
