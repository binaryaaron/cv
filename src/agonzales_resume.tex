\DocumentMetadata{
 testphase=phase-II,
 pdfversion=2.0
}


\documentclass[print]{ag-cv} % Add 'print' as an option into the square bracket to remove colors from this template for printing
%%%%%%%%%%%%%%%%%%%%%%%%%% %%%%%%%%%%%%%%%%%%%%%%%%%% %%%%%%%%%%%%%%%%%%%%%%%%%%
%%%%%%%%%%%%%%%%%%%%%%%%%% %%%%%%%%%%%%%%%%%%%%%%%%%% %%%%%%%%%%%%%%%%%%%%%%%%%%
%%%%%%%%%%%%%%%%%%%%%%%%%% %%%%%%%%%%%%%%%%%%%%%%%%%% %%%%%%%%%%%%%%%%%%%%%%%%%%
%%%%%%%%%%%%%%%%%%%%%%%%%% %%%%%%%%%%%%%%%%%%%%%%%%%% %%%%%%%%%%%%%%%%%%%%%%%%%%

% \usepackage{microtype}
\usepackage{fontspec}
\defaultfontfeatures{Mapping=tex-text,Ligatures={NoRequired,NoCommon}}



\addbibresource{agbib.bib} % Specify the bibliography file to include publications


%%%%%%%%%%%%%%%%%%%%%%%%%% %%%%%%%%%%%%%%%%%%%%%%%%%% %%%%%%%%%%%%%%%%%%%%%%%%%%
%%%%%%%%%%%%%%%%%%%%%%%%%% %%%%%%%%%%%%%%%%%%%%%%%%%% %%%%%%%%%%%%%%%%%%%%%%%%%%
%%%%%%%%%%%%%%%%%%%%%%%%%% %%%%%%%%%%%%%%%%%%%%%%%%%% %%%%%%%%%%%%%%%%%%%%%%%%%%
%%%%%%%%%%%%%%%%%%%%%%%%%% %%%%%%%%%%%%%%%%%%%%%%%%%% %%%%%%%%%%%%%%%%%%%%%%%%%%
\title{Aaron Gonzales}
\author{Aaron Gonzales}

\makeatletter

\renewcommand\maketitle{
{\raggedright % Note the extra {
{\smallheaderfont aaron@aarongonzales.net~|~505-385-9209~|~github.com/binaryaaron}\\
\vspace{3mm}
{\headingfont \bfseries \@title }
}
} % Note the extra }
\makeatother
\usepackage[document]{ragged2e}

\begin{document}

\pagenumbering{gobble} % removes page numbers
\pagestyle{fancy} % removes header/footer
\fancyhf{}
\fancyhead{}
\fancyfoot{}
\cfoot{}
\renewcommand{\headrulewidth}{0pt}
\renewcommand{\footrulewidth}{0pt}
\maketitle

\raggedright
\section*{Selected Experience}

\begin{job}
  {Staff Machine Learning Engineer}
  {Mozilla.ai}
  {April 2023}
  {current}
  {Remote, USA}
  {
  Hired as Lead/Founding Engineer to develop Mozilla.ai's foundational research and development platform.
  }
{
  \begin{myitemize}
    \item Established engineering culture and practice, including coding
    standards, developer tooling, CI/CD pipelines, API design styles,
    abstraction patterns, and a technical hiring program. 
    \item Designed and built a platform for large language model (LLM) research
    and product iteration on a cloud-based GPU cluster, enabling early
    participation and sponsorship of Neurips and KDD workshops \autocite{kdd_2023_workshop,neurips_llm_efficiency_2023}. 
    \item Onboarded and mentored four ML engineers and researchers. 
    \item Manage procurement, setup, and relationships of cloud and IT vendors.
  \end{myitemize}
}
\end{job}


\begin{job}
  {Senior Machine Learning Research Engineer}
  {Twitter}
  {June 2020}
  {January 2023}
  {Remote, USA}
  {
    Technical Lead for the Machine Learning Ethics, Transparency, and Accountability team that owned
    Responsible Machine Learning development and research at Twitter.
  }
  {
    \begin{myitemize}
      \item Improved processes, gave technical guidance, mentored other engineers, and led organization-wide improvements.

      \item Built a system of feature pipelines and tools for researchers that accelerated critical team research by 20x while ensuring more reliable results.

      \item Led projects to build tools into Twitter's internal Tensorflow Extended ecosystem that enabled teams to rapidly identify undue biases and weaknesses in model performance.

      \item Facilitated adoption of team results and products, leading to company-wide adoption of model quality metrics and tools. Presented work in multiple forums \autocite{twitter_htl_racial_bias,twitter_recsys_distributional_inequality,facct_2022_talk,datascience_2022_talk,privacy_enchancing_tech_post}.

      \item Continued service on company advisory groups, including efforts to select a new Twitter-wide data processing stack that would lead to 3-5x speedups in key data pipelines.
    \end{myitemize}
  }
\end{job}

\begin{job}
  {Senior Data Scientist}
  {Twitter}
  {July 2017}{June 2020}
  {Boulder, CO}
  {
  Data scientist for Twitter's API client business and later as tech lead for the Health Data Science team. 
  }
{
    \begin{myitemize}
      \item Established tools and libraries that enabled ~50 data scientists and
      ML engineers to increase the scale of analyses by 5-50x by making PySpark,
      Presto, and BigQuery easier to use and scale internally on disparate
      datasets.
      
      \item Developed strategy, model, and data pipeline for identifying
      potentially compromised accounts.
      
      \item Developed open-source client libraries for Twitter API products, which
      were used by thousands of API customers
      \autocite{search_tweets_python,do_more_with_twitter_data}.
      
      \item Built risk model for developers applying to use Twitter's APIs, and
      model for identifying clusters of applications abusing Twitter's API
      policies, allowing programmatic remediation of thousands of abusive client
      applications.
      
      \item Served on company advisory groups for data architecture, open-source
      software, Python development, and as a global lead for Twitter's
      Hispanic/Latinx employees resource group.
    \end{myitemize}
    } 

\end{job}


\begin{job}
  {Data Scientist}
  {Tripadvisor}
  {July 2016}{July 2017}
  {Greater Boston Area, MA}
  {
  Built scalable machine learning models for the Vacation Rentals group.
  }
  {
  \begin{myitemize}
    \item Shipped a computer vision feature-engineering pipeline with models to
    detect humans, extract content, and rate aesthetic quality for all photos,
    increasing a click-through rate metric by 4.8\% and increasing revenue from
    ad expenditures by 10.4%.
    
    \item Built a language model and interface that enables natural-language
    queries to find collections of destinations, which led to a 6.1\%
    improvement in a primary metric.
    
    \item Developed language-modeling tools to extract high-information snippets
    from reviews to display on search-result pages to boost SEO rankings.
    
    \item Developed an internal data and model tools package that allowed data
    scientists and engineers to get up-to-speed more quickly and iterate faster
    on projects.
\end{myitemize}
}
\end{job}

%----------------------------------------------------------------------------------------
%	Education
%----------------------------------------------------------------------------------------

\section*{Education}
  \begin{edu}{Master of Science, Computer Science}{2016}{University of New Mexico}
  \end{edu}
  \begin{edu}{Bachelor of Science, Psychology}{2010}{University of New Mexico}
  \end{edu}

% \section*{Technical Skills}

%   \begin{skilltype}
%     {Languages}{Python, SQL, shell, Scala}
%   \end{skilltype}
%   \begin{skilltype}
%     {ML/Data/AI frameworks}{Spark, Tensorflow, Kubeflow, TFX, Pytorch, Pandas, scikit-learn, Keras, vllm, Ray}
%   \end{skilltype}

%   \begin{skilltype}
%     {Dev and ML Ops frameworks/platforms}{Google Cloud, Coreweave, Oracle Cloud, Kubernetes, Helm, Airflow, Bazel}
%   \end{skilltype}
\begin{skills}
  {Languages}{Python, SQL, shell, Scala}
  {ML/Data/AI frameworks}{Spark, Tensorflow, Kubeflow, TFX, Pytorch, Pandas, scikit-learn, Keras, vllm, Ray}
  {Dev and ML Ops frameworks/platforms}{Google Cloud, Coreweave, Oracle Cloud, Kubernetes, Helm, Airflow, Bazel}
\end{skills}


\section*{Selected Publications and Artifacts}
  
  %%%%%%%%%%%%% . else non CV
  \newrefcontext[sorting=chronological]
  \nocite{network_induced_memory_contention,
    % comp_brain_phenotyping_paper,
    skullstrip,do_more_with_twitter_data} \leavevmode\printbibliography[heading=none,keyword={selected}] 
    

\end{document}
