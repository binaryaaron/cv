%%%%%%%%%%%%%%%%%%%%%%%%%%%%%%%%%%%%%%%%%
% Friggeri Resume/CV  - modified by Aaron Gonzales
% XeLaTeX Template
%
% Original author:
% Adrien Friggeri (adrien@friggeri.net)
% https://github.com/afriggeri/CV
%
% License:
% CC BY-NC-SA 3.0 (http://creativecommons.org/licenses/by-nc-sa/3.0/)
%
% Important notes:
% This template needs to be compiled with XeLaTeX and the bibliography, if used,
% needs to be compiled with biber rather than bibtex.
%
% Aaron Gonzales
% changed many formatting options, did away with tabular as a style, ditched
% sidebar, changed fonts, margins for the 2nd+ page, etc.
%%%%%%%%%%%%%%%%%%%%%%%%%%%%%%%%%%%%%%%%%

\documentclass[print]{friggeri-cv} % Add 'print' as an option into the square bracket to remove colors from this template for printing

\addbibresource{agbib.bib} % Specify the bibliography file to include publications
\usepackage{marvosym}
\usepackage{fontspec}
% \defaultfontfeatures{Path = /usr/local/texlive/2018/texmf-dist/fonts/opentype/public/fontawesome/}
% \usepackage{fontawesome}
\usepackage{fancyhdr}
\usepackage{lastpage}
\RequirePackage[quiet]{fontspec}
% \RequirePackage{unicode-math}

% \newfontfamily{\FAFR}{Font Awesome 5 Free Regular}
% \newfontfamily{\FAFS}{Font Awesome 5 Free Solid}
% \newfontfamily{\FAB}{Font Awesome 5 Brands Regular}
% \def\faEmail{{\FAFR \symbol{"F0E0}}} % Email
% \def\faPhone{\FAFS \symbol{"F095}} % Phone
% \def\faLinkedin{\FAB \symbol{"F08C}} % Linkedin
% \def\faGithub{\FAB \symbol{"F09B}} % Github
% \def\faStackOverflow{\FAB \symbol{"F16C}} % StackOverflow

\setmainfont[Mapping=tex-text, Color=textcolor]{Helvetica Neue Light}
\newfontfamily\bodyfont[]{Hack}
\newfontfamily\thinfont[]{Hack}
\newfontfamily\headingfont[]{Futura}
\newfontfamily\largeheaderfont[Color=textcolor, Scale=1.15]{Futura}
\newfontfamily\smallheaderfont[Color=textcolor, Scale=0.9]{Futura}

\rfoot{}
%\cfoot{\thinfont{Gonzales CV; page \thepage}}
\cfoot{}
\lhead{}
\chead{}
\rhead{}

\newcommand{\tripicon}{\includegraphics[scale=0.05]{trip_logo.jpg}}%
\newcommand{\twittericon}{\includegraphics[scale=0.05]{Twitter_Logo_Blue.png}}%

%%%%% added packages by me
%\newbibmacro*{name:bold}[2]{\iffieldequalstr{hash}{a924b9d23960e50ebbd52c9af2c9dd44}{\bfseries}{}}
\usepackage{etoolbox}
\newtoggle{cv}

% block to take doc args
\ifx\docmode\undefined
\togglefalse{cv}
\else 
  \toggletrue{cv}
\fi
\begin{document}

% \newgeometry{ left=1.75cm,top=2cm,right=1.75cm,bottom=1.50cm,nohead,nofoot }
\iftoggle{cv}
{\header{Aaron}{~Gonzales}{curriculum vit\ae}
  {\{\href{mailto:aaron@aarongonzales.net}{aaron@aarongonzales.net},
  505.385.9209,
  \href{http://aarongonzales.net}{aarongonzales.net}\}}
}
{\header{Aaron}{~Gonzales}
  {\{\href{mailto:aaron@aarongonzales.net}{aaron@aarongonzales.net},
  505.385.9209,
  \href{http://aarongonzales.net}{aarongonzales.net}\}}
  % {~\href{http://lnkd.in/b8kfQSe}{\Large{\faLinkedin}},
  % \href{http://github.com/binaryaaron}{\huge{\faGithub} }}
}

%----------------------------------------------------------------------------------------
%	SIDEBAR SECTION
%----------------------------------------------------------------------------------------

% \begin{aside} % In the aside, each new line forces a line break
% \section{contact}
% \href{mailto:agonzales@cs.unm.edu}{agonzales@cs.unm.edu}
% \href{http://aarongonzales.net}{aarongonzales.net}
% 505.385.9209
% %1332 Vassar NE
% %Albuquerque, NM 87106
% \href{http://lnkd.in/b8kfQSe}{\scriptsize{\faLinkedin}\ }, \href{http://github.com/xysmas}{\scriptsize{\faGithub} }
% %\href{http://lnkd.in/b8kfQSe}{\scriptsize{\faLinkedin} \ \ \normalsize{aaronknowsdata} }
% %\href{http://github.com/xysmas}{\scriptsize{\faGithub} \ \ \normalsize{xysmas}}
% ~
% %\section{languages}
% \section{programming at a glance}
% Java, Python, R
% ~
% \section{about}
% Computer science graduate student with strong analytical skills and extensive
% research experience seeking a data science position after May 2016 graduation.
% \end{aside}

% \section{about}
% Computer science graduate student with strong analytical skills and extensive
% research experience seeking a data science position after May 2016 graduation.

%----------------------------------------------------------------------------------------
%         Research/work	
%----------------------------------------------------------------------------------------
% \section{about}
{
  \section{Summary}
  Data and machine learning professional with a blend of statistics, machine
  learning, engineering, research, and leadership skills who is comfortable
  working across the machine-learning lifecycle.  }
% I solve problems.
\iftoggle{cv}
{\section{Experience}
}
{\section{Selected Experience}
}

\begin{description} \itemsep1pt \parskip0pt \parsep0pt
  \item \twittericon {\largeheaderfont Senior Machine Learning Research Engineer, Twitter } \hfill
    {\smallheaderfont June 2020 \textemdash  current}\\*
    {\footnotesize \emph{Remote - USA}} \\*
    Tech Lead for the Machine Learning Ethics, Transparency, and Accountability
    (META) team tasked with Responsible Machine Learning development and
    research at Twitter.
    % {\smallheaderfont Notable accomplishments}:
    \begin{itemize} \itemsep1pt \parskip1pt \parsep0pt

    \item  Served as a technical lead and provided process improvement,
    technical guidance, roadmap planning, and organization-wide improvements. 
    
    \item Designed and built a data collection system that would eventually
    automate team measurement and impact projects going forward. Reduced time
    for future similar research projects from months of manual effort to < 1
    week
    
    \item Led workstream to design and build tools for machine learning
    practitioners to identify weaknesses in model performance that have been
    used internally by multiple teams
    
    \item Presented at conferences (ACM FAccT, The Data Science Conference)
    about building and scaling an applied ML research team
    
    \item Contributes to company working groups around Python development, data
    architecture, and Latinx employee support.
  \end{itemize}
  \end{description}

\begin{description} \itemsep1pt \parskip0pt \parsep0pt
  \item \twittericon {\largeheaderfont Senior Data Scientist, Twitter } \hfill
    {\smallheaderfont May 2019 \textemdash  June 2020}\\*
    {\footnotesize \emph{Boulder, CO}} \\*
    \begin{itemize} \itemsep1pt \parskip1pt \parsep0pt

      \item Tech Lead for the Health data science team, which focuses on on
      anti-abuse, spam, fake accounts, and misinformation problems at Twitter.
      
      \item Built a model to detected accounts that are potentially compromised
      
      \item As tech lead, provided process improvement, technical guidance, roadmap
      planning, and organization-wide improvements.
      
      \item Formed team to develops tooling, education, processes, and relationships
      with peer teams to make doing data science at Twitter better for all.
      
      \item Served as a Global Lead for Twitter's Alas BRG for Hispanic and Latinx
      employees, serve on company-wide organizations, and as a co-chair for our
      Latinx engineers group at Twitter.
      
  \end{itemize}

  \end{description}

\begin{description} \itemsep1pt \parskip0pt \parsep0pt
  \item \twittericon {\largeheaderfont Data Scientist, Twitter } \hfill
    {\smallheaderfont July 2017 \textemdash  May 2019}\\*
    {\footnotesize \emph{Boulder, CO}} \\*
    Worked with the Developer and Enterprise Solutions group to help improve
    Twitter’s developer platform, vet new product ideas, and make datasets that
    enabled teams to generate insights and make decisions quickly.
    
    % {\smallheaderfont Notable accomplishments}:
    \begin{itemize} \itemsep1pt \parskip1pt \parsep0pt
      \item Developed an open-source python API for our premium and enterprise search
          products (see the Search Tweets Python link to Github for more info) which has been downloaded and used 
          many thousands of times.
      \item Led project to provide real-world examples using Twitter data ton
      and support developers and data scientists to build their own solutions
      with Twitter data products. See the Do more with Twitter data series links
      for more info.
      \item Identified clusters of applications abusing Twitter's APIs using time-series mining
      \item Identified risk factors in developer applications to enable programmatic
            review of applications to use Twitter APIs.
    \end{itemize}
\end{description}

\begin{description} \itemsep1pt \parskip0pt \parsep0pt
  \item \tripicon {\largeheaderfont Data Scientist, TripAdvisor } \hfill
    {\smallheaderfont July 2016\textemdash July 2017}\\*
    {\footnotesize \emph{Greater Boston Area, MA}} \\*
    % {\smallheaderfont Notable accomplishments}:
    \begin{itemize} \itemsep1pt \parskip1pt \parsep0pt
      \item Designed and built a computer vision system that includes models to
      detect people, content (e.g., beach, bedroom, kitchen), and rate aesthetic
      quality.

      \item Computer vision system lead to immediate 4.8\% increases in primary
      click-through rate and a 10.4\% increase in return on ad spend. It was
      broadly adopted for other purposes across TripAdvisor.

      \item Built an app that enables natural-language queries (e.g. ”girls getaway”)
      to find collections of destinations, which led to a 6.1\% improvement in a
      primary metric and was used by many teams. 

      \item Developed methods to extract high-information snippets from reviews
      to display on search-result pages to boost SEO rankings.

      \item Developed a common team codebase providing shared tools, common
      access methods, and other utility code that allowed our data scientists
      and engineers to get up-to-speed more quickly and iterate faster on
      projects.  \end{itemize}
  \end{description}

\begin{description} \itemsep1pt \parskip0pt \parsep0pt
  \item \tripicon {\largeheaderfont Data Scientist Intern, TripAdvisor } \hfill
    {\smallheaderfont Summer 2015}\\*
    {\footnotesize \emph{Greater Boston Area, MA}} \\*
    Data scientist intern with the Vacation Rentals team.
    % {\smallheaderfont Notable accomplishments}:
    Developed a novel ranking system for
    720,000+ rental properties and used gradient-boosting machines to predict
    how well new properties will perform. \\*
    \begin{itemize} \itemsep1pt \parskip1pt \parsep0pt
      \item The model scaled with millions of daily visitors and self-tunes to
            fluctuations in visitor usage patterns
      \item A/B testing showed that the model \textbf{increased a key visitor conversion
            rate by 3.46\%} and \textbf{decreased visitor bounce rate by 0.46\%}
      \item A/B testing showed that integrating the model into rankings helped
            \textbf{increase revenue per visitor by 9.57\%}
      \item The model was put into production across all of TripAdvisor Vacation Rentals' sites\\*
    \end{itemize}

  \iftoggle{cv}{
  \item {\largeheaderfont Research Assistant, The University of New Mexico} \hfill
    {\smallheaderfont 2015\textemdash 2016}\\*
    {\footnotesize \emph{Albuquerque, NM }} \\*
    Research assistant for Dorian Arnold, PhD, in the Scalable Systems Lab. We
    partnered with both Los Alamos National Laboratory and the Center for
    Advanced Research Computing to investigate applying data science techniques
    to understand complex high-performance system behavior. \\*
  }{}

  \item {\largeheaderfont Analyst/Programmer, The University of New Mexico} \hfill 
    {\smallheaderfont 2011\textemdash 2014 }\\*
    {\footnotesize \emph{Albuquerque, NM }} \\*
    {\smallheaderfont summary}: \\*
    Lead analyst in a neuroscience research lab run by Elaine Bearer,
    MD\textemdash PhD, managing various research projects and lab members. \\*
    % {\smallheaderfont Notable accomplishments}:
    \begin{itemize} \itemsep0.5pt \parskip0pt \parsep0pt
      \item Streamlined lab data processing and analytical techniques, including a
        method that speed up a critical data processing step by approximately 360x
        (3 hours to 30 seconds). Implemented numerous other batch data processing
        steps for other tasks.
      \item Designed and conducted a pilot study that helped win a \$2.7 million
        dollar NIH R01 grant to study the etiology of post traumatic stress
        disorder.
      \item Lead author on several research papers, three conference abstracts
      and presentations, coauthor on many more submitted and pending papers and
      abstracts
      \item Trained and mentored 10 undergraduate and gradute student
        employees and volunteers
    \end{itemize}

%------------------------------------------------
\iftoggle{cv}
{\item  {\largeheaderfont Research Associate} \hfill {\smallheaderfont 2010\textemdash 2011} \\*
    {\footnotesize \emph{The Mind Research Network, Albuquerque, NM}}  \\*
    {\smallheaderfont summary:} \\*
    RA in a neuroscience research group ran by Julia Stephen, PhD. Contributed
    to a study investigating multi-sensory integration in patients with
    schizophrenia. \\*
\end{description}}
{\end{description}}
%\newpage
%------------------------------------------------
%\section{research internships}
  \iftoggle{cv}{
\begin{description}
  \item  {\largeheaderfont Student Volunteer } {\smallheaderfont \hfill 2008\textemdash 2010} \\*
    {\footnotesize \emph{The Mind Research Network, Albuquerque, NM}} \hfill \\*
    Assisted with a study ran by Pilar Sanjuan, PhD investigating substance use
    and post-traumatic stress disorder (PTSD) in recently returned combat
    veterans. \\*
\end{description}

%%%%%%% sets this to remove the sidebar %%%%%%%%%%
%\newgeometry{ left=1.75cm,top=2cm,right=1.75cm,bottom=1.50cm,nohead,nofoot }

\begin{description}
  \item  {\largeheaderfont Student Volunteer}  \hfill {\smallheaderfont 2008 } \\*
    {\footnotesize \emph{The University of New Mexico, Albuquerque, NM}\textemdash } \\*
    Assisted Akaysha Tang, PhD, with a study investigating stress regulation in
    rats and assisted an expert in troubleshooting and repairing an EEG system
    for a study investigating stress in humans.\\*
%------------------------------------------------
\end{description}
}{}
%----------------------------------------------------------------------------------------
%	Education
%----------------------------------------------------------------------------------------

\section{Education}
\begin{description}
%------------------------------------------------
  \item {\largeheaderfont Master of Science, Computer Science}  \hfill 
    {\smallheaderfont 2016} \\*
    {\footnotesize \emph{The University of New Mexico  \hfill 3.7 cumulative GPA}} \\*
    Concentration in data mining and machine learning.
%------------------------------------------------
\end{description}
\begin{description}
%------------------------------------------------
  \item  {\largeheaderfont Bachelor of Science, Psychology } \hfill
    {\smallheaderfont 2010 }\\*
    {\footnotesize \emph{The University of New Mexico  }} \\*
    Concentration in neuroscience; minored in computer science.
%------------------------------------------------
\end{description}

%------------------------------------------------
% \newpage
% \section{Technical Skills}
% \begin{description}
%    \item {\smallheaderfont Programming languages, notable libraries, and tools} \\*
%         \textbf{R}, Python (scikit-learn, gensim,
%         Matplotlib, Pandas, Statsmodels, Cython, Sqlalchemy, Keras), Spark, C, Bash,
%         \LaTeX, git, svn, MongoDB, Hadoop, Hive, SQL (Postgres,
%         MSSQL). Some experience with Scala, Java, Javascript, HTML, CSS,
%         Matlab, and Amazon Web Services (EC2, S3, Redshift).
%     \item {\smallheaderfont Machine Learning / Data Science Methods} \\*
%         Deep learning, supervised learning (random forests, gradient boosting,
%         regression, SVMs), unsupervised learning
%         (autoencoders, DBSCAN, k-means/mediods, EM), unstructured data, web
%         scraping outlier analysis, novelty detection, time-series mining,
%         dimensionality reduction, and feature selection.
%     \item {\smallheaderfont Visualization and Miscellaneous tools} \\*
%         Bokeh, Matplotlib, ggplot, shiny, and d3.js.

% \end{description}


\iftoggle{cv}{
\section{Publications}
}
{%\newgeometry{ left=1.75cm,top=2cm,right=1.75cm,bottom=1.50cm,nohead,nofoot }
\section{Selected Publications}
}

\iftoggle{cv}{
\begin{refsection} % This is a custom heading for those references marked as "inproceedings" but not containing "keyword=france"
\nocite{*}

\newrefcontext[sorting=chronological]
\printbibliography[type=article, title={Articles In Peer-Reviewed Journals}, heading=subbibliography]

\newrefcontext[sorting=chronological]
\printbibliography[type=inproceedings, title={Conference Proceedings}, heading=subbibliography]

% \newrefcontext[sorting=chronological]
% \printbibliography[type=misc, title={Manuscripts in Preparation}, heading=subbibliography]
\end{refsection}
}
%else non CV
{%\begin{refsection} % This is a custom heading for those references marked as "inproceedings" but not containing "keyword=france"
%\newgeometry{ left=1.75cm,top=2cm,right=1.75cm,bottom=1.50cm,nohead,nofoot }

\begin{refsection}
\nocite{*}
\newrefcontext[sorting=chronological]
\printbibliography[type=article, title={Articles In Peer-Reviewed Journals}, heading=subbibliography]
\end{refsection}

\begin{refsection}
\nocite{*}
\newrefcontext[sorting=chronological]
\printbibliography[type=inproceedings, title={Selected Conference Proceedings}, keyword={selected}, heading=subbibliography]
\end{refsection}
}

%----------------------------------------------------------------------------------------
%	INTERESTS SECTION
%----------------------------------------------------------------------------------------

\section{Other Interests and Accolades}
\textbf{Olympic Weightlifting}
\begin{itemize}
\item 2014 New Mexico Games: Gold Medalist, 94kg class
\item 2013 New Mexico Games: Silver Medalist, 85kg class
\item 2013 Barnholth Memorial Invitational: Silver Medalist, 85kg class
\end{itemize}

\end{document}
