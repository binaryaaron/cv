%%%%%%%%%%%%%%%%%%%%%%%%%%%%%%%%%%%%%%%%%
% Friggeri Resume/CV  - modified by Aaron Gonzales
% XeLaTeX Template
%
% Original author:
% Adrien Friggeri (adrien@friggeri.net)
% https://github.com/afriggeri/CV
%
% License:
% CC BY-NC-SA 3.0 (http://creativecommons.org/licenses/by-nc-sa/3.0/)
%
% Important notes:
% This template needs to be compiled with XeLaTeX and the bibliography, if used,
% needs to be compiled with biber rather than bibtex.
%
% Aaron Gonzales
% changed many formatting options, did away with tabular as a style, ditched
% sidebar, changed fonts, margins for the 2nd+ page, etc.
%%%%%%%%%%%%%%%%%%%%%%%%%%%%%%%%%%%%%%%%%

\documentclass[print]{friggeri-cv} % Add 'print' as an option into the square bracket to remove colors from this template for printing

\addbibresource{agbib.bib} % Specify the bibliography file to include publications
\usepackage{marvosym}
\usepackage{fontspec}
% \defaultfontfeatures{Path = /usr/local/texlive/2018/texmf-dist/fonts/opentype/public/fontawesome/}
% \usepackage{fontawesome}
\usepackage{fancyhdr}
\usepackage{lastpage}
\RequirePackage[quiet]{fontspec}
% \RequirePackage{unicode-math}

% \newfontfamily{\FAFR}{Font Awesome 5 Free Regular}
% \newfontfamily{\FAFS}{Font Awesome 5 Free Solid}
% \newfontfamily{\FAB}{Font Awesome 5 Brands Regular}
% \def\faEmail{{\FAFR \symbol{"F0E0}}} % Email
% \def\faPhone{\FAFS \symbol{"F095}} % Phone
% \def\faLinkedin{\FAB \symbol{"F08C}} % Linkedin
% \def\faGithub{\FAB \symbol{"F09B}} % Github
% \def\faStackOverflow{\FAB \symbol{"F16C}} % StackOverflow

\setmainfont[Mapping=tex-text, Color=textcolor]{Helvetica Neue Light}
\newfontfamily\bodyfont[]{Hack}
\newfontfamily\thinfont[]{Hack}
\newfontfamily\headingfont[]{Futura}
\newfontfamily\largeheaderfont[Color=textcolor, Scale=1.15]{Futura}
\newfontfamily\smallheaderfont[Color=textcolor, Scale=0.9]{Futura}

\rfoot{}
%\cfoot{\thinfont{Gonzales CV; page \thepage}}
\cfoot{}
\lhead{}
\chead{}
\rhead{}

\newcommand{\tripicon}{\includegraphics[scale=0.05]{trip_logo.jpg}}%
\newcommand{\twittericon}{\includegraphics[scale=0.05]{Twitter_Logo_Blue.png}}%

%%%%% added packages by me
%\newbibmacro*{name:bold}[2]{\iffieldequalstr{hash}{a924b9d23960e50ebbd52c9af2c9dd44}{\bfseries}{}}
\usepackage{etoolbox}
\newtoggle{cv}
% \togglefalse{cv}
\toggletrue{cv}
\begin{document}

% \newgeometry{ left=1.75cm,top=2cm,right=1.75cm,bottom=1.50cm,nohead,nofoot }
\iftoggle{cv}
{\header{Aaron}{~Gonzales}{curriculum vit\ae}}
{\header {Aaron}{~Gonzales}
{\{\href{mailto:aaron@aarongonzales.net}{aaron@aarongonzales.net},
505.385.9209,
\href{http://aarongonzales.net}{aarongonzales.net}\}}
  % {~\href{http://lnkd.in/b8kfQSe}{\Large{\faLinkedin}},
  % \href{http://github.com/binaryaaron}{\huge{\faGithub} }}
}

%----------------------------------------------------------------------------------------
%	SIDEBAR SECTION
%----------------------------------------------------------------------------------------

% \begin{aside} % In the aside, each new line forces a line break
% \section{contact}
% \href{mailto:agonzales@cs.unm.edu}{agonzales@cs.unm.edu}
% \href{http://aarongonzales.net}{aarongonzales.net}
% 505.385.9209
% %1332 Vassar NE
% %Albuquerque, NM 87106
% \href{http://lnkd.in/b8kfQSe}{\scriptsize{\faLinkedin}\ }, \href{http://github.com/xysmas}{\scriptsize{\faGithub} }
% %\href{http://lnkd.in/b8kfQSe}{\scriptsize{\faLinkedin} \ \ \normalsize{aaronknowsdata} }
% %\href{http://github.com/xysmas}{\scriptsize{\faGithub} \ \ \normalsize{xysmas}}
% ~
% %\section{languages}
% \section{programming at a glance}
% Java, Python, R
% ~
% \section{about}
% Computer science graduate student with strong analytical skills and extensive
% research experience seeking a data science position after May 2016 graduation.
% \end{aside}

% \section{about}
% Computer science graduate student with strong analytical skills and extensive
% research experience seeking a data science position after May 2016 graduation.

%----------------------------------------------------------------------------------------
%         Research/work	
%----------------------------------------------------------------------------------------
% \section{about}
% I solve problems.
\iftoggle{cv}
{\section{Experience}
}
{\section{Selected Experience}
}

\begin{description} \itemsep1pt \parskip0pt \parsep0pt
  \item \twittericon {\largeheaderfont Senior Data Scientist, Twitter } \hfill
    {\smallheaderfont May 2019 \textemdash  current}\\*
    {\footnotesize \emph{Boulder, CO}} \\*
    Working on Account Security related work to detect and measure the
    incidence of compromise on our platform.

    Served as a technical lead and provided process improvement, technical
    guidance, roadmap planning, and data-science organization wide
    improvements. Mentored colleagues. Reviews work for the greater Health data
    science team, working on anti-abuse, spam, fake account, and misinformation
    problems at Twitter.

    Also serve as the technical lead for data science effectiveness, a virtual
    team that develops tooling, processes, and relationships with peer teams
    to make doing data science at Twitter better for all.

    Serve as a Global Lead for Twitter's
    Alas BRG for Hispanic and Latinx employees, serve on company-wide
    organizations, and as a co-chair for our Latinx engineers group at Twitter.

  \end{description}

\begin{description} \itemsep1pt \parskip0pt \parsep0pt
  \item \twittericon {\largeheaderfont Data Scientist, Twitter } \hfill
    {\smallheaderfont July 2017 \textemdash  May 2019}\\*
    {\footnotesize \emph{Boulder, CO}} \\*
    Worked with the Developer and Enterprise Solutions group to help improve our
    developer platform, vet new product ideas, and make datasets that enabled
    teams to generate insights and make decisions quickly.

    {\smallheaderfont Notable accomplishments}:
    \begin{itemize} \itemsep1pt \parskip1pt \parsep0pt
      \item Developed an open-source python API for our premium and enterprise search
          products (see the Search Tweets Python link to Github for more info. As of
          April 2019, the library has been installed over 18,000 times and has > 100
          stars on github.
      \item Managed developer education project to provided real-world example analyses
        using Twitter data to lower barriers to adoption and support developers \& data
        scientists to build their own solutions with Twitter data products. See the Do
        more with Twitter data series links for more info.
      \item Identified clusters of applications abusing Twitter's APIs using time-series mining
      \item Identified risk factors in developer applications to enable programmatic
            review of applications to use our APIs.
    \end{itemize}
  \end{description}

\begin{description} \itemsep1pt \parskip0pt \parsep0pt
  \item \tripicon {\largeheaderfont Data Scientist, TripAdvisor } \hfill
    {\smallheaderfont July 2016\textemdash July 2017}\\*
    {\footnotesize \emph{Greater Boston Area, MA}} \\*
    Data scientist for the Vacation Rentals group.
    {\smallheaderfont Notable accomplishments}:
    \begin{itemize} \itemsep1pt \parskip1pt \parsep0pt
      \item Designed and built a computer vision pipeline for all photos on the Rentals
        platform using deep learning methods. Includes models built to
        detect people (with and without faces), content (e.g., beach, bedroom,
        kitchen), and aesthetic quality. Using the
        data from the pipeline, Display marketing saw a 4.8\% increase in click-through
        rate and a 10.4\% increase in ROAS.
      \item Built an internal app that supports natural-language
        queries (e.g. "girls getaway") to find collections of destinations using
        deep-learning word embedding techniques. Used across teams 
        to help find data-driven destinations, dynamically improve relevance
        (A/B tested with 6.1\% improvement over control in a primary metric), 
        and to recommend nearby destinations for travelers.
      \item Designed and built a modeling pipeline to predict property availability,
        assisting our product team's ability to encourage better traveler experiences
        and owner behavior.
      \item Developed methods to extract high-information snippets from reviews to
        display on search result pages to boost SEO rankings.
      \item Developed a common team codebase providing shared tools, common access
        methods, and other utility code that is in heavy use by the team. It has
        allowed our data scientists and engineers to get up-to-speed more quickly and
        iterate faster on projects.
    \end{itemize}
  \end{description}

\begin{description} \itemsep1pt \parskip0pt \parsep0pt
  \item \tripicon {\largeheaderfont Data Scientist Intern, TripAdvisor } \hfill
    {\smallheaderfont Summer 2015}\\*
    {\footnotesize \emph{Greater Boston Area, MA}} \\*
    Data scientist intern with the Vacation Rentals team.
    {\smallheaderfont Notable accomplishments}:
    Developed a novel ranking system for
    720,000+ rental properties and used gradient-boosting machines to predict
    how well new properties will perform. \\*
    \begin{itemize} \itemsep1pt \parskip1pt \parsep0pt
      \item The model was built with approximately 5 terabytes of web traffic
            history using Hive, Python, pandas, and scikit-learn
      \item The model scales with millions of daily visitors and self-tunes to
            fluctuations in visitor usage patterns
      \item A/B testing showed that the model \textbf{increased a key visitor conversion
            rate by 3.46\%} and \textbf{decreased visitor bounce rate by 0.46\%}
      \item A/B testing showed that integrating the model helped
            \textbf{increase revenue per visitor by 9.57\%}
      \item The model was put into production across all of TripAdvisor Vacation Rentals' sites\\*
    \end{itemize}

  \item {\largeheaderfont Research Assistant, The University of New Mexico} \hfill
    {\smallheaderfont 2015\textemdash 2016}\\*
    {\footnotesize \emph{Albuquerque, NM }} \\*
    Research assistant for Dorian Arnold, PhD, in the Scalable Systems Lab. We
    partnered with both Los Alamos National Laboratory and the Center for
    Advanced Research Computing to investigate applying data science techniques
    to understand complex high-performance system behavior. \\*

  \item {\largeheaderfont Analyst/Programmer, The University of New Mexico} \hfill 
    {\smallheaderfont 2011\textemdash 2014 }\\*
    {\footnotesize \emph{Albuquerque, NM }} \\*
    {\smallheaderfont summary}: \\*
    Lead analyst in a neuroscience research lab run by Elaine Bearer,
    MD\textemdash PhD, managing various research projects and lab members. \\*
    {\smallheaderfont Notable accomplishments}:
    \begin{itemize} \itemsep0.5pt \parskip0pt \parsep0pt
      \item Streamlined lab data processing and analytical techniques, including a
        method that speed up a critical data processing step by approximately 360x
        (3 hours to 30 seconds). Implemented numerous other batch data processing
        steps for other tasks.
      \item Designed and conducted a pilot study that helped win a \$2.7 million
        dollar NIH R01 grant to study the etiology of post traumatic stress
        disorder.
      \item Lead author on five research papers (one submitted, four nearing
        submission), three conference abstracts and presentations, coauthor on many
        more submitted and pending papers and abstracts
      \item Trained and mentored 10 undergraduate and post-baccalaureate student
        employees and volunteers
    \end{itemize}

%------------------------------------------------
\iftoggle{cv}
{\item  {\largeheaderfont Research Associate} \hfill {\smallheaderfont 2010\textemdash 2011} \\*
    {\footnotesize \emph{The Mind Research Network, Albuquerque, NM}}  \\*
    {\smallheaderfont summary:} \\*
    RA in a neuroscience research group ran by Julia Stephen, PhD. Contributed
    to a study investigating multi-sensory integration in patients with
    schizophrenia. \\*
\end{description}}
{\end{description}}
%\newpage
%------------------------------------------------
%\section{research internships}
  \iftoggle{cv}{
\begin{description}
  \item  {\largeheaderfont Student Volunteer } {\smallheaderfont \hfill 2008\textemdash 2010} \\*
    {\footnotesize \emph{The Mind Research Network, Albuquerque, NM}} \hfill \\*
    Assisted with a study ran by Pilar Sanjuan, PhD investigating substance use
    and post-traumatic stress disorder (PTSD) in recently returned combat
    veterans. \\*
\end{description}

%%%%%%% sets this to remove the sidebar %%%%%%%%%%
%\newgeometry{ left=1.75cm,top=2cm,right=1.75cm,bottom=1.50cm,nohead,nofoot }

\begin{description}
  \item  {\largeheaderfont Student Volunteer}  \hfill {\smallheaderfont 2008 } \\*
    {\footnotesize \emph{The University of New Mexico, Albuquerque, NM}\textemdash } \\*
    Assisted Akaysha Tang, PhD, with a study investigating stress regulation in
    rats and assisted an expert in troubleshooting and repairing an EEG system
    for a study investigating stress in humans.\\*
%------------------------------------------------
\end{description}
}{}
%----------------------------------------------------------------------------------------
%	Education
%----------------------------------------------------------------------------------------

\section{Education}
\begin{description}
%------------------------------------------------
  \item {\largeheaderfont Master of Science, Computer Science}  \hfill 
    {\smallheaderfont 2016} \\*
    {\footnotesize \emph{The University of New Mexico  \hfill 3.7 cumulative GPA}} \\*
    Concentration in data mining and machine learning.
%------------------------------------------------
\end{description}
\begin{description}
%------------------------------------------------
  \item  {\largeheaderfont Bachelor of Science, Psychology } \hfill
    {\smallheaderfont 2010 }\\*
    {\footnotesize \emph{The University of New Mexico  }} \\*
    Concentration in neuroscience; minored in computer science.
%------------------------------------------------
\end{description}

%------------------------------------------------
% \newpage
% \section{Technical Skills}
% \begin{description}
%    \item {\smallheaderfont Programming languages, notable libraries, and tools} \\*
%         \textbf{R}, Python (scikit-learn, gensim,
%         Matplotlib, Pandas, Statsmodels, Cython, Sqlalchemy, Keras), Spark, C, Bash,
%         \LaTeX, git, svn, MongoDB, Hadoop, Hive, SQL (Postgres,
%         MSSQL). Some experience with Scala, Java, Javascript, HTML, CSS,
%         Matlab, and Amazon Web Services (EC2, S3, Redshift).
%     \item {\smallheaderfont Machine Learning / Data Science Methods} \\*
%         Deep learning, supervised learning (random forests, gradient boosting,
%         regression, SVMs), unsupervised learning
%         (autoencoders, DBSCAN, k-means/mediods, EM), unstructured data, web
%         scraping outlier analysis, novelty detection, time-series mining,
%         dimensionality reduction, and feature selection.
%     \item {\smallheaderfont Visualization and Miscellaneous tools} \\*
%         Bokeh, Matplotlib, ggplot, shiny, and d3.js.

% \end{description}


\iftoggle{cv}{
\section{Publications}
}
{%\newgeometry{ left=1.75cm,top=2cm,right=1.75cm,bottom=1.50cm,nohead,nofoot }
\section{Publications}
}

\iftoggle{cv}{
\begin{refsection} % This is a custom heading for those references marked as "inproceedings" but not containing "keyword=france"
\nocite{*}

\newrefcontext[sorting=chronological]
\printbibliography[type=article, title={Articles In Peer-Reviewed Journals}, heading=subbibliography]

\newrefcontext[sorting=chronological]
\printbibliography[type=inproceedings, title={Conference Proceedings}, heading=subbibliography]

% \newrefcontext[sorting=chronological]
% \printbibliography[type=misc, title={Manuscripts in Preparation}, heading=subbibliography]
\end{refsection}
}
%else non CV
{%\begin{refsection} % This is a custom heading for those references marked as "inproceedings" but not containing "keyword=france"
%\newgeometry{ left=1.75cm,top=2cm,right=1.75cm,bottom=1.50cm,nohead,nofoot }

\begin{refsection}
\nocite{*}
\newrefcontext[sorting=chronological]
\printbibliography[type=article, title={Articles In Peer-Reviewed Journals}, heading=subbibliography]
\end{refsection}

\begin{refsection}
\nocite{*}
\newrefcontext[sorting=chronological]
\printbibliography[type=inproceedings, title={Selected Conference Proceedings}, keyword={selected}, heading=subbibliography]
\end{refsection}
}

%----------------------------------------------------------------------------------------
%	INTERESTS SECTION
%----------------------------------------------------------------------------------------

\section{Other Interests and Accolades}
\textbf{Olympic Weightlifting}
\begin{itemize}
\item 2014 New Mexico Games: Gold Medalist, 94kg class
\item 2013 New Mexico Games: Silver Medalist, 85kg class
\item 2013 Barnholth Memorial Invitational: Silver Medalist, 85kg class
\end{itemize}

\end{document}
