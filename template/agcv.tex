%%%%%%%%%%%%%%%%%%%%%%%%%%%%%%%%%%%%%%%%%
% Friggeri Resume/CV 
% XeLaTeX Template
% Version 1.0 (5/5/13)
%
% This template has been downloaded from:
% http://www.LaTeXTemplates.com
%
% Original author:
% Adrien Friggeri (adrien@friggeri.net)
% https://github.com/afriggeri/CV
%
% License:
% CC BY-NC-SA 3.0 (http://creativecommons.org/licenses/by-nc-sa/3.0/)
%
% Important notes:
% This template needs to be compiled with XeLaTeX and the bibliography, if used,
% needs to be compiled with biber rather than bibtex.
% 
% Aaron Gonzales
% Added formatting and other options
%%%%%%%%%%%%%%%%%%%%%%%%%%%%%%%%%%%%%%%%%

\documentclass[]{friggeri-cv} % Add 'print' as an option into the square bracket to remove colors from this template for printing

\addbibresource{agbib.bib} % Specify the bibliography file to include publications
\usepackage{marvosym}
\usepackage{fontspec}
\usepackage{fontawesome}
\usepackage{fancyhdr}
\usepackage{lastpage}
\RequirePackage[quiet]{fontspec}
\RequirePackage[math-style=TeX,vargreek-shape=unicode]{unicode-math}

\newfontfamily\bodyfont[]{Helvetica Neue}
\newfontfamily\thinfont[]{Helvetica Neue UltraLight}
\newfontfamily\headingfont[]{Helvetica Neue Condensed Bold}

\rfoot{\thinfont{Gonzales CV; page \thepage}}
\cfoot{}
\lhead{}
\chead{}
\rhead{}



%%%%% added packages by me
\newbibmacro*{name:bold}[2]{\iffieldequalstr{hash}{a924b9d23960e50ebbd52c9af2c9dd44}{\bfseries}{}}

\begin{document}

\header{aaron}{gonzales}{curriculum vit\ae} % Your name and current job title/field % removed title

%----------------------------------------------------------------------------------------
%	SIDEBAR SECTION
%----------------------------------------------------------------------------------------

\begin{aside} % In the aside, each new line forces a line break
\section{contact}
505.750.2214 
%505.750.2214
1332 Vassar NE
Albuquerque, NM 87106
~
\href{mailto:agonzales@cs.unm.edu}{agonzales@cs.unm.edu}
~
%\href{http://www.smith.com}{http://www.smith.com}
\href{http://lnkd.in/b8kfQSe}{Linked\scriptsize{\faLinkedin}}
%\section{languages}
%english mother tongue
\section{programming at a glance}
Java, Python, R, C, Bash
~
\section{about}
I enjoy using Computer Science to solve emergent problems in our world.
\end{aside}



%----------------------------------------------------------------------------------------
%	WORK EXPERIENCE SECTION
%----------------------------------------------------------------------------------------

\section{education}
\begin{entrylist}
%------------------------------------------------
\entry
{2016}
{Master of Science {\normalfont in Computer Science}}
{University of New Mexico}
{Specializing in Data Mining and Visualization}
%------------------------------------------------
\end{entrylist}

\begin{entrylist}
%------------------------------------------------
\entry
{2010}
{Bachelor of Science {\normalfont in Psychology}}
{University of New Mexico}
{Minor in Computer Science}
%------------------------------------------------
\end{entrylist}

%----------------------------------------------------------------------------------------
%	COMPUTER SKILLS SECTION
%----------------------------------------------------------------------------------------
\section{research/work experience}

\begin{entrylist}
%------------------------------------------------
\entry
{\footnotesize{2011--2014}}
%{Elaine Bearer, MD\textendash PhD \textemdash University of New Mexico}
%{Albuquerque, NM}
%%%%%%%%%%%%%%%%%%%%%%suggested from nina lanza%
{University of New Mexico, Albuquerque, NM \textemdash Supervisor: Elaine Bearer, MD\textendash PhD}
{}
{\emph{\textbf{Analyst/Programmer}} \\
Senior analyst in a neuroscience research lab in the School of Medicine 
working on and managing various projects. \\
\textbf{primary duties}: \\
Analyzing scientific data, primarily manganese-enhanced magnetic resonance imaging (MEMRI) data 
taken from transgenic mice,
vesicular cargo transport in axons, 
DNA methylation, and histological microscopy. Analysis methods included
statistical parametric mapping, 
extensions of the general linear model, 
data visualization, and modeling.
Writing programs and scripts to automate or facilitate analysis 
that can be used by other lab members for analysis on their projects. 
Implementing reproducible research methods and documentation. 
Developing methods to expedite research capacity. \\
\textbf{other duties}: \\
Preparing scientific posters, presentations, and manuscripts, 
assisting with grant writing, database searching, and study design. 
Training, instructing, and managing both student employees and volunteers at various levels of training 
(e.g., early undergraduate, late undergraduate, postbaccalaureate, and graduate). }
\end{entrylist}
%------------------------------------------------

\begin{entrylist}
\entry
{\footnotesize{2010--2011}}
%{Julia Stephen, PhD \textemdash The Mind Research Network}
%{Albuquerque, NM}
{The Mind Research Network, Albuquerque, NM \textemdash Supervisor: Julia Stephen, PhD}
{}
{\emph{\textbf{Research Associate}} \\
Worked on a study investigating multi-sensory integration in
patients with schizophrenic and healthy normal volunteers. \\
\textbf{duties}:\\
Acquiring data by scanning research subjects using both functional
magnetic resonance imaging (fMRI) and magnetoencephalography (MEG), writing
programs and scripts to automate analysis tasks, 
participant payment, data analysis.} \\ \\ 
\end{entrylist}

%------------------------------------------------
\section{research internships}
\begin{entrylist}
\entry
{\footnotesize{2008-2010}}
%{Pilar Sanjuan, PhD \textemdash The Mind Research Network}
%{Albuquerque, NM}
{The Mind Research Network, Albuquerque, NM \textemdash Supervisor: Pilar Sanjuan, PhD}
{}
{\emph{\textbf{Student Volunteer}} \\
Assisted with a study investigating substance use disorders and
post-traumatic stress disorder (PTSD) in recently returned combat veterans. \\
\textbf{duties}: \\
Recruiting, entering data, constructing SPSS databases, scoring
measures, analyzing data, editing fMRI and MEG tasks in Presentation; running research
participants in fMRI, running participants in the MEG scanner; phone screening participants,
some analysis of fMRI image data; tracking and reviewing pre-processing of imaging data.}
\end{entrylist}

\begin{entrylist}
\entry
{\footnotesize{2008}}
{University of New Mexico, Albuquerque, NM \textemdash Supervisor: Akaysha Tang, PhD}
{}
{\textbf{\emph{Student Volunteer}} \\
Assisted with a study investigating stress regulation in rats and assisted an
expert in troubleshooting and repairing an EEG system for a study investigating stress in
humans. \\
\textbf{duties}: \\
Collecting, coding, and entering data, gathering literature for a grant
proposal, testing and analyzing EEG equipment, contacting equipment vendors, replacing
computer components, diagnosing and rebuilding workstations.}
%------------------------------------------------
\end{entrylist}

%------------------------------------------------
%------------------------------------------------
%------------------------------------------------
\section{consulting work}
\begin{entrylist}
\entry
{\footnotesize{2014}}
{Pima Vascular}
{}
{\textbf{\emph{Statistical Consultant}} \\
Analyzed data from a clincical trial for a novel stent used in vascular surgery. 
Analysis included data munging, visualization, Cox Proportional Hazard modeling, 
and reproducible reporting using \textbf{R}.}
\end {entrylist}

\begin{entrylist}
\entry
{\footnotesize{2014}}
{Fame4good.com}
{}
{\textbf{\emph{Social Media Consultant}} \\
Tested and analyzed advertisements as they affected goal conversions and website traffic.}
\end {entrylist}

%----------------------------------------------------------------------------------------
%	EDUCATION SECTION
%----------------------------------------------------------------------------------------

\section{computer skills}
\begin{description}
   \item[\textbf{\textit{programming languages and tools}}] \hfill \\  
	 MATLAB, Java, \textbf{R}, C, Python, bash, Knitr, \LaTeX, git
   %\item[\textbf{\textit{editors and IDEs}}] \hfill \\
	 %Eclipse, RStudio, vim
   \item[\textbf{\textit{software packages and operating systems}}] \hfill \\
	 %\begin{description} \itemsep1pt \parskip0pt \parsep0pt
	   %\item[software packages] \hfill \\
	    \textbf{General} \\
		Linux (Ubuntu/CentOS), Microsoft Windows, Apple OS X \\
	   %\item[specialized] \hfill \\
	    \textbf{Specialized} \\
	    Adobe Illustrator, Adobe InDesign, Adobe Photoshop, Amira, ImageJ, MIPAV, SPM8, FSL, NiftyReg, MetaMorph 
	 %\end{description}
\end{description}


%----------------------------------------------------------------------------------------
%	PUBLICATIONS SECTION
%----------------------------------------------------------------------------------------

\section{publications}

\printbibsection{article}{articles in peer-reviewed journals} % Print all articles from the bibliography

%\printbibsection{book}{books} % Print all books from the bibliography

\begin{refsection} % This is a custom heading for those references marked as "inproceedings" but not containing "keyword=france"
\nocite{*}
\printbibliography[sorting=chronological, type=inproceedings, title={conference proceedings}, notkeyword={france}, heading=subbibliography]
\end{refsection}

%\begin{refsection} % This is a custom heading for those references marked as "inproceedings" and containing "keyword=france"
%\nocite{*}
%\printbibliography[sorting=chronological, type=inproceedings, title={local peer-reviewed conferences/proceedings}, keyword={france}, heading=subbibliography]
%\end{refsection}


%\printbibsection{report}{research reports} % Print all research reports from the bibliography

\printbibsection{misc}{manuscripts in preparation} % Print all miscellaneous entries from the bibliography


%----------------------------------------------------------------------------------------
%----------------------------------------------------------------------------------------
%	Relevant coursework SECTION
%----------------------------------------------------------------------------------------
\section{relevant coursework}
\begin{description}
   \item[\textbf{\textit{statistics}}] \hfill \\
	 Advanced Data Analysis I - Graduate statistics with a focus on practical analysis using \textbf{R}. 
	 Topics included linear regression, non-parametric statistics, literate programming, and general linear methods. \hfill  \\
	 Advanded Data Analysis II - Continuation of ADAI.
	 Topics include cluster analysis, principal component analysis, multivariate methods, 
	 logistic and polynomial regression, experimental design, visualization, and efficient code using \textbf{R}.
   \item[\textbf{\textit{computer science}}] \hfill \\
	 %\begin{itemize}
	   Computational Linguistics – semester project involved four persons 
	   working on a Java-based joke generator querying a substantial SQL database using 
	   XML templates with Baysien inference and other machine-learning features to improve quality 
	   of the jokes. \hfill \\
	   Data Structures\hfill \\
	   Algorithms \hfill \\
	   Software Engineering \hfill \\
	   Artificial Intelligence (graph searching algorithims, including $A^*$, Dijkstra). 
	 %\end{itemize}
   \item[\textbf{\textit{psychology}}] \hfill \\  
	 Neuroimaging, Research Methods 
%   \item[\textbf{\textit{other}}] \hfill \\
%	 Linear Algebra \textemdash Khan Academy; MIT OpenCourseware \hfill \\
	 %Introduction to Data Science \textemdash University of Washington via Coursera \hfill \\
%	 Algorithms: Design and Analysis, Part 1 \textemdash Stanford University via Coursera
\end{description}

%----------------------------------------------------------------------------------------
%	INTERESTS SECTION
%----------------------------------------------------------------------------------------

\section{other interests/accolades}


\textbf{personal} \\
Olympic Weightlifting 
\begin{itemize}
\item 2014 New Mexico Games: Gold Medalist, 94kg class
\item 2013 New Mexico Games: Silver Medalist, 85kg class
\item 2013 Barnholth Memorial Invitational: Silver Medalist, 85kg class
\end{itemize}
UNM Mountaineering Club President; 2007-2008 \\

\end{document}
